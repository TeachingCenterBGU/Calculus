% ==============================
% Dummy Overleaf Project (HE) — Linear Algebra Book
% Exact environments: definition, example, exercise, remark,
% theorem, lemma, proposition, corollary, proof, solution, hint
% Numbering by chapter; solution/hint unnumbered.
% ==============================
\documentclass[12pt]{book}

% ---- Hebrew + fonts ----
\usepackage{fontspec}
\usepackage{polyglossia}
\setdefaultlanguage{hebrew}
\setotherlanguage{english}
% Overleaf has Frank Ruehl CLM in TeX Live
\newfontfamily\hebrewfont[Script=Hebrew]{Frank Ruehl CLM}
\newfontfamily\hebrewfontsf[Script=Hebrew]{Frank Ruehl CLM}
\newfontfamily\hebrewfonttt[Script=Hebrew]{Frank Ruehl CLM}

% ---- Math & friends ----
\usepackage{amsmath,amssymb,mathtools}
\usepackage{amsthm}
\usepackage{hyperref}
\usepackage{graphicx}
\usepackage{booktabs}
\usepackage{enumitem}
\usepackage{tikz}
\usepackage{pgfplots}

% ---- Theorem-like environments (by chapter) ----
\theoremstyle{plain}
\newtheorem{theorem}{משפט}[chapter]
\newtheorem{lemma}[theorem]{למה}
\newtheorem{proposition}[theorem]{טענה}
\newtheorem{corollary}[theorem]{מסקנה}

\theoremstyle{definition}
\newtheorem{definition}[theorem]{הגדרה}
\newtheorem{example}[theorem]{דוגמה}
\newtheorem{exercise}[theorem]{תרגיל}
\newtheorem{remark}[theorem]{הערה}

% Unnumbered for website folding
\newtheorem*{solution}{פתרון}
\newtheorem*{hint}{רמז}

% ---- Macros (optional) ----
\newcommand{\R}{\mathbb{R}}

\title{ספר אלגברה לינארית — דגם אוברליף}
\author{צוות הקורס}
\date{\today}

\begin{document}
\frontmatter
\maketitle
\tableofcontents
\mainmatter

\chapter{מרחבים וקטוריים}
\selectlanguage{hebrew}
\section{סדרות}
% ✅ Now using a real environment

יתכן מאוד וכבר שמעתם על סדרות חשבוניות,
לדוגמא,
$1,2,3,4,5,\ldots$, 
וסדרות הנדסיות, 
לדוגמא, 
$1,\frac{1}{2},\frac{1}{4},\frac{1}{8},\frac{1}{16},\ldots$. 
אלה הן שתי משפחות של סדרות של מספרים.		 
\begin{definition}
{\it סדרה של מספרים ממשיים} 
היא פונקציה 
$f$ 
אשר תחום הגדרתה הוא קבוצת המספרים הטבעיים וטווחה הוא קבוצת המספרים הממשיים. 
נסכים לכתוב 
$a_n$ 
במקום 
$f(n)$.
\end{definition}
\[
\begin{array}{cccccc}
1 & 2 & 3 & \ldots & n & \ldots \\
\downarrow & \downarrow & \downarrow & \downarrow & \downarrow & \downarrow \\
f(1) & f(2) & f(3) & \ldots & f(n) & \ldots
\end{array}
\]


כמו לכל פונקציה, גם לסדרה יש הצגה גרפית. לכל מספר סידורי 
$n$ 
על ציר ה- 
$x$ 
נתאים 
נקודה 
$(n,a_n)$. 
לדוגמא, הסדרה שמתוארת באיור היא 
$a_n=\frac{1}{n}$. 


\vspace{1cm}

% Plot
\selectlanguage{english}
\begin{center}
\begin{tikzpicture}
  \begin{axis}[
    axis lines = left,
    xlabel = {$n$},
    ylabel = {$a_n=\frac{1}{n}$},
    xtick = {1,2,3,4,5,6,7},
    ytick = {0, 0.2, 0.4, 0.6, 0.8, 1.0},
    ymin=0, ymax=1.1,
    xmin=0.5, xmax=7.5,
    width=12cm,
    height=6cm,
    grid=both,
  ]
    \addplot[only marks, mark=*, mark size=2pt] coordinates {
      (1,1)
      (2,0.5)
      (3,0.333)
      (4,0.25)
      (5,0.2)
      (6,0.166)
      (7,0.142)
    };
  \end{axis}
\end{tikzpicture}
\end{center}

\selectlanguage{hebrew}
נהוג לסמן סדרה ע"י 
$\{a_n\}_{n=1}^\infty$, 
כש- 
$a_{17}$ 
הוא האיבר ה- 
$17$ 
שלה ו- 
$a_{129}$ 
הוא האיבר ה- 
$129$ 
שלה. 
לעתים בכתיבת איברי הסדרה מוותרים על הסוגריים ורושמים פשוט 
$a_1,a_2,a_3,\ldots$. 

\begin{example}\rm
    \begin{enumerate}
         \item 
$\{n\}_{n=1}^\infty$ 
היא סדרה של מספרים טבעיים המופיעים לפי הסדר ה"טבעי" שלהם, כלומר,
$$1,2,3,4,5,\ldots.$$
\item $\{2n-1\}_{n=1}^\infty$ 
היא סדרה של מספרים טבעיים אי-זוגיים, 
כלומר, 
$$1,3,5,7,9,\ldots.$$
\item הסדרה 
$a_n=1$ 
לכל 
$n$, 
כלומר 
$$1,1,1,1,1,\ldots.$$
\item $\{(-1)^n\}_{n=1}^\infty$ 
היא סדרה של מספרים 
$-1$ 
ו- 
$1$ 
המתחלפים, כלומר, 
$$-1,1,-1,1,-1,\ldots.$$
\item $\left\{\frac{(-1)^{n+1}}{n}\right\}_{n=1}^\infty$ 
היא הסדרה הבאה 
$$1,-\frac{1}{2},\frac{1}{3},-\frac{1}{4},\frac{1}{5},\ldots.$$
    \end{enumerate}
\end{example}



קיימים אופנים שונים להגדרת סדרה. בכל אחת מהדוגמאות לעיל, סדרה הוגדרה ע"י נוסחא לחישוב איברה הכללי. 
זהו לא האופן היחיד. 
אחת השיטות הנפוצות להגדרת סדרה היא השיטה 
{\it רקורסיבית} - 
השיטה שמאפשרת חישוב של איברה הכללי של סדרה דרך האיברים הקודמים. בשיטה זאת נחוצים 
{\it תנאיי התחלה}. 
לדוגמא, 
תהא 
$\{a_n\}_{n=1}^\infty$ 
מוגדרת ע"י 
$a_{n}=a_{n-2}+a_{n-1}$ 
כאשר 
$a_1=1$
ו- 
$a_2=1$. 
)שימו לב כי סדרות רקורסיביות עם נוסחת נסיגה זהה שנבדלות בתנאיי התחילה הן סדרות שונות.(
לא קשה להשתכנע כי איבריה הראשונים של הסדרה הם
$1,1,2,3,5,8,13,21,\ldots$. 
קיימת גם נוסחא לחישוב של איברה הכללי של הסדרה הזאת
$$a_n=\frac{(1+\sqrt{5})^n-(1-\sqrt{5})^n}{2^n\sqrt{5}}.$$
זאת סדרה "ידוענית" בתחום המתמטיקה, היא נקראת 
{\it סדרת פיבונצ'י}, 
על כינויו של מתמטיקאי איטלקי מימי הביניים- לאונרדו פיזנו.

\begin{theorem}
אם \( a > b \) ו־\( b > c \), אז \( a > c \).
\end{theorem}

\end{document}